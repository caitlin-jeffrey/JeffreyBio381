%%%%%%%%%%%%%%%%%%%%%%%%%%%%%%%%%%%%%%%%%
% Beamer Presentation
% LaTeX Template
% Version 1.0 (10/11/12)
%
% This template has been downloaded from:
% http://www.LaTeXTemplates.com
%
% License:
% CC BY-NC-SA 3.0 (http://creativecommons.org/licenses/by-nc-sa/3.0/)
%
% Modified by Nicholas J. Gotelli
% 9 January 2021
%%%%%%%%%%%%%%%%%%%%%%%%%%%%%%%%%%%%%%%
\documentclass[12pt]{beamer}
% only 10,11, or 12 pt fonts
% PACKAGES-----------------------------------
\usepackage{graphicx} % Allows including images
\usepackage{booktabs} % Allows the use of \toprule, \midrule and \bottomrule in tables

% THEMES AND COLORS-------------------------
\mode<presentation> {
\usefonttheme{default}
% FONTTHEMES: default, structurebold, structuresmallcapsserif, structureitalicserif, serif, professionalfonts


\usetheme{Montpellier}
% THEMES: default, AnnArbor, Antibes, Bergen, Berkeley, Berlin, Boadilla, boxes, CambridgeUS, Copenhagen, Darmstadt, Dresden, Frankfurt, Goettingen, Hannover, Ilmenau, JuanLesPins, Luebeck, Madrid, Malmoe, Marburg, Montpellier, PaloAlto, Pittsburgh, Rochester, Singapore, Szeged, Warsaw

\usecolortheme{rose}
%COLORTHEMES: default, albatross, beaver, beetle, crane, dolphin, dove, fly, lily, orchid, rose, seagull, seahorse, sidebartab, structure, whale, wolverine 

% DISPLAY OPTIONS--------------------------
%\setbeamertemplate{footline} % To remove the footer line in all slides, uncomment this line

%\setbeamertemplate{footline}[page number] % To replace the footer line in all slides with a simple slide count, uncomment this line

%\setbeamertemplate{navigation symbols}{} % To remove the navigation symbols from the bottom of all slides, uncomment this line
}
% -----------------------------------------

% TITLE PAGE DATA--------------------------
\title[Beamer talk]{Let's try out this Beamer thing: My first presentation} % The short title appears at the bottom of every slide, the full title is only on the title page

\author{Caitlin E. Jeffrey} % Your name

\institute[UVM] % Your institution as it will appear on the bottom of every slide, may be shorthand to save space
{
University of Vermont \\ % Your institution for the title page
Department of Animal and Veterinary Science \\
Burlington, VT 05401 USA \\ 
\medskip
\textit{caitlin.jeffrey@uvm.edu} % Your email address
}
\date{24 February 2021} % Date, can be changed to a custom date or \today
% -----------------------------------------

% BEGIN DOCUMENT---------------------------
\usepackage{Sweave}
\begin{document}
\Sconcordance{concordance:Homework_04_Beamer.tex:Homework_04_Beamer.Rnw:%
1 58 1 1 0 184 1}


% OPTIONAL TITLE PAGE SLIDE----------------
\begin{frame}
\titlepage % Print the title page as the first slide
\end{frame}

% OPTIONAL TABLE OF CONTENTS SLIDE---------

% \begin{frame}
% \frametitle{Overview} % Table of contents slide, comment this block out to remove it
% \tableofcontents % Throughout your presentation, if you choose to use \section{} and \subsection{} commands, these will automatically be printed on this slide as an overview of your presentation
% \end{frame}
% 
% % OPTIONAL SECTION HEADERS-----------------
% \section{First Section} % Sections can be created in order to organize your presentation into discrete blocks; all sections and subsections are automatically printed in the table of contents as an overview of the talk
% 
% \subsection{Subsection Example} % A subsection can be created just before a set of slides with a common theme to further break down your presentation into chunks

% SLIDE (BULLET POINTS)--------------------
% \begin{frame}
% \frametitle{Bullet Points}
% \begin{itemize}
% \item first item
% \item second item
% \item third item
% \item et cetera
% \item last item
% \end{itemize}
% \end{frame}

% SLIDE (SEQUENTIAL BULLET POINTS)---------
\begin{frame}
\frametitle{Objectives:}
\begin{itemize}
\item<1-> Explore the diagnostic ability of quarter-level somatic cell testing in diagnosing intramammary infections caused by CNS on organic dairies in Vermont
\item<2-> Explore the diagnostic ability of quarter-level somatic cell testing in diagnosing intramammary infections caused by Corynebacterium species on organic dairies in Vermont
%\item<3-> Text visible on third slide
\end{itemize}
\end{frame} 

% SLIDE (FIGURE)-----------------------------
\begin{frame}
\frametitle{ANATOMY LESSON:}
% Uncomment the code on this slide to include your own image from the same directory as the template  file.
% \begin{figure}
   \includegraphics[width=1.0\linewidth]{ducklette.jpg}
   \caption{This is a ducklette}
% use this format for absolute sizing
%\includegraphics[width=3cm, height=4cm]{filename.jpg}
% \end{figure}
\end{frame}

% SLIDE (FIGURE)-----------------------------
\begin{frame}
\frametitle{Can I fit 2 images?}
% Uncomment the code on this slide to include your own image from the same directory as the template  file.
% \begin{figure}
   \includegraphics[height=0.35\textheight]{cherf_algonquin.jpg}
   \caption{...sure}
   \hfill
   \includegraphics[height=0.35\textheight]{wright_view.jpg}
% use this format for absolute sizing
%\includegraphics[width=3cm, height=4cm]{filename.jpg}
% \end{figure}
\end{frame}




% SLIDE (TABLE)----------------------------
\begin{frame}
\frametitle{SCC by housing type}
\begin{table}
\begin{tabular}{l l l}
\toprule
\textbf{Farm Number} & \textbf{Facility type} & \textbf{Avg. SCC}\\
\midrule
Farm 1 & Tiestall & 71,000 \\
Farm 2 & Tiestall & 118,000 \\
Farm 3 & Bedded pack & 142,000 \\
Farm 4 & Bedded pack & 235,000 \\
\bottomrule
\end{tabular}
\caption{Bedded packs had higher average SCC at study start}
\end{table}
\end{frame}

%------------------------------------------------
\section{Second Section}
%------------------------------------------------
% % SLIDE (PARAGRAPHS OF TEXT)---------------
% \begin{frame}
% \frametitle{Paragraphs of Text}
% Sed iaculis dapibus gravida. Morbi sed tortor erat, nec interdum arcu. Sed id lorem lectus. Quisque viverra augue id sem ornare non aliquam nibh tristique. Aenean in ligula nisl. Nulla sed tellus ipsum. Donec vestibulum ligula non lorem vulputate fermentum accumsan neque mollis.\\~\\
% 
% Sed diam enim, sagittis nec condimentum sit amet, ullamcorper sit amet libero. Aliquam vel dui orci, a porta odio. Nullam id suscipit ipsum. Aenean lobortis commodo sem, ut commodo leo gravida vitae. Pellentesque vehicula ante iaculis arcu pretium rutrum eget sit amet purus. Integer ornare nulla quis neque ultrices lobortis. Vestibulum ultrices tincidunt libero, quis commodo erat ullamcorper id.
% \end{frame}

% SLIDE (BLOCKS OF HIGHLIGHTED TEXT)-------
\begin{frame}
\frametitle{How did CNS and Corynebacterium prevalence compare to previous work?}
\begin{block}{CNS subclinical infection avg. prevalence 19.3\%}
Our average prevalence for CNS was around 19 percent. Previous work has found that prevalence seems to vary widely from herd to herd or geographic area.
\end{block}

\begin{block}{Corynebacterium spp. subclinical infection avg. prevalence 2.2\%}
For Corynebacterium, we found an average prevalence of around 2 percent. Again, prevalence for this pathogen seems variable by herd, but this fits into the range found between two farms in a large study in the Northeast.
\end{block}
\end{frame}

% % SLIDE (EMBEDDED R CODE)------------------
% \begin{frame}[fragile]{Embedded R Code; \texttt{fragile} frame}
% \begin{block}
% 
% <<>>=
% # show some output...
% runif(10)
% @
% 
% \end{block}
% \end{frame}

% % SLIDE (EMBEDDED R FIGURE)----------------
% \begin{frame}[fragile]{Embedded R Figure; \texttt{fragile} frame}
% %\begin{block}
% 
% %<<fig.align='center',fig.dim=c(2.5,2.5),echo=FALSE>>=
% # limited space for output
% plot(runif(10))
% @
% 
% %\end{block}
% \end{frame}

% % SLIDE (MULTIPLE COLUMNS)-----------------
% \begin{frame}
% \frametitle{Multiple Columns}
% \begin{columns}[c] % The "c" option specifies centered vertical alignment while the "t" option is used for top vertical alignment
% 
% \column{.45\textwidth} % Left column and width
% \textbf{Heading}
% \begin{enumerate}
% \item Statement
% \item Explanation
% \item Example
% \end{enumerate}
% 
% \column{.5\textwidth} % Right column and width
% Lorem ipsum dolor sit amet, consectetur adipiscing elit. Integer lectus nisl, ultricies in feugiat rutrum, porttitor sit amet augue. Aliquam ut tortor mauris. Sed volutpat ante purus, quis accumsan dolor.
% 
% \end{columns}
% \end{frame}
% 
% 
% % SLIDE (THEOREM)----------------------------
% \begin{frame}
% \frametitle{Theorem}
% \begin{theorem}[Mass--energy equivalence]
% $E = mc^2$
% \end{theorem}
% \end{frame}
% 
% % SLIDE (VERBATIM)---------------------------
% \begin{frame}[fragile] % Need to use the fragile option when verbatim is used in the slide
% \frametitle{Verbatim}
% \begin{example}[Theorem Slide Code]
% \begin{verbatim}
% \begin{frame}
% \frametitle{Theorem}
% \begin{theorem}[Mass--energy equivalence]
% $E = mc^2$
% \end{theorem}
% \end{frame}\end{verbatim}
% \end{example}
% \end{frame}

% SLIDE (FINAL SLIDE)------------------------
\begin{frame}
\Huge{\centerline{Fin}}
\end{frame}

%------------------------------------------------
\end{document}
